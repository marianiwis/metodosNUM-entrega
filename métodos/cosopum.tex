% ======== Plantilla para elaboracion de informes ========
% version: 3.2  - Junio/2017 
% autores:  J. M. Pérez Zerpa - P. Curto - M. Caballero
% Comentarios y sugerencias en el foro del taller:
%   https://eva.fing.edu.uy/course/view.php?id=307
%
% ----- LICENCIA -------------------------------------------------------------
% Este trabajo es distribuido bajo la licencia LaTeX Project Public License
% Puede y debe ser usado, modificado y/o distribuido gratuitamente.
% ----------------------------------------------------------------------------
%
% --- Comentarios ---
% Durante todo el archivo se ven comentarios escritos con porcentaje al inicio
% dado que LaTeX es un lenguaje de programacion compilado, los comentarios
% no tienen valor ninguno para el compilador pero si para el programador (humano). Esto puede ser útil para agregar descripciones para el usuario mientras elabora un documento o si es realizado entre varias personas.
%
% ============================================================================

% --- Clase de archivo ---
% Con este comando se define que tipo de documento vamos a hacer
% en este caso un articulo, en una hoja a4 y con tamanio de fuente 11pt.
\documentclass[a4paper,11pt]{article}
% los tamanios mas utilizados son 10,11 y 12 pt y representan un tamanio de referencia, o sea que al cambiar a 12 pt todos los textos van a aumentar proporcionalmente, al estilo homotecia.

% Luego de definir la clase de documento se definen paquetes o funciones a ser utilizadas dentro del documento.


% --- codificacion de archivo ---
% El paquete inputenc sirve para que el compilador pueda interpretar los acentos en espaniol u otros idiomas, dependiendo de la codificacion del archivo indicada en la parte inferior de texstudio en la mayoria de los casos se usa UTF8, pero tambien son utilizados otros formatos por defecto en funcion de la configuracion del editor y el sistema operativo. Se debe incluir solamente una de estas opciones
%
% este archivo esta codificado como utf8 (opcion por defecto del editor texstudio):
\usepackage[utf8]{inputenc}
% otros formatos posibles son
%\usepackage[ansinew]{inputenc}
%\usepackage[latin1]{inputenc}
%\usepackage[applemac]{inputenc}

% ----------------------------------

% --- idioma ---
% El paquete babel sirve para indicar a latex el idioma en que se deben generar ciertos campos de forma automática como la fecha, los títulos de figuras, índice etc. Por ejemplo "Indice" en lugar de "Contents".
\usepackage[spanish]{babel} 
% --------------


% separar correctamente las palabras en muchos idiomas, aquií­ esta configurado con espaniol (spanish). Tambien sirve para que el Ií­ndic etenga tií­tulo "
\usepackage{hyphenat}

% La fuente utilizada por defecto en latex es llamada Knuth's Computer Modern, %
% Si quitamos el comentario de alguna de las opciones de abajo utilizaremos otra fuente hay muchas otras opciones en http://www.tug.dk/FontCatalogue/ . 
%\usepackage{sans}
%\usepackage{fbb}
%\usepackage{bera}
%\usepackage{sans}
%\usepackage{times}
%\usepackage{libertine} 


% --- Gráficos ---
% El paquete graphicx sirve para controlar figuras con \includegraphics.
\usepackage{graphicx}
% Esta linea indica el lugar (path) en el cual se encuentra la carpeta donde colocamos imágenes, para ahorrar colocarlo en cada
% llamado de una imagen en el documento.
\graphicspath{{./figuras/}}

% Este paquete permite colocar varias figuras en el entorno "figure" como subfiguras.
\usepackage{subcaption} 

% --- Lenguaje matemático ---
% fuentes para escribir sí­mbolos
\usepackage{amsfonts,amssymb,amsthm,amsmath}


% --- Tablas ---
% Paquetes para el manejo de tablas, creación de filas y columnas unidas.
\usepackage{multirow} 
\usepackage{multicol} 
% Control de color en tablas muy versátil.
\usepackage[table]{xcolor}

% --- Hipervínculos ---
% paquete para marcar los hiperví­nculos en i­ndice y referencias
\usepackage{hyperref}
% Para citar referencias  
\usepackage{cite} 
\hypersetup{colorlinks, urlcolor=cyan, citecolor=green, linkcolor=blue} % Pinta con color las referencias
\usepackage[hypcap]{caption} % Las imágenes tienen hiperreferencia y se ven completas y no solo la leyenda. 
% Para hacer hiperreferencias a páginas web
\usepackage{url} 

\usepackage{booktabs} 

% Para agregar al índice las refencias
\usepackage[nottoc,notlot,notlof]{tocbibind} 

% Aqui comienzan los datos del trabajo. El comando \date{\today} asigna la fecha en que se compila como la fecha del trabajo, tambien se puede escibe directamente, ej. \date{5 de setiembre de 2012}.

% estos datos son utilizados unicamente si se genera el título con el comando \maketitle (ver mas abajo)
\title{Obligatorio 1}
\author{%
  Guadalupe Sánchez - 5.537.458-1\\ %
  Ignacio Pérez - 5.298.266-0\\ %
  Mariana Álvarez - 5.254.138-7\\ %
  Tiagui Samurio - 5.240.686-6\\ %
  Grupo 34 - Tutor: Juliana Faux \\ %
  Obligatorio $1$ - Métodos Numéricos 2022.}
\date{\today} % aqui se puede incluir otra fecha
% ------------------------


% ====================================


%El paquete colortbl sirve para darle color a las tablas
\usepackage{colortbl}

% Este paquete se utiliza para generar texto de relleno.
%\usepackage{blindtext}


% ===== Encabezado =====
% esta es una posible configuración para el encabezado. 
%Si se comentan estas dos lineas no habrá encabezado y la numeración de página aparecerá abajo de cada hoja. En la página donde se llame a \maketitle no se coloca encabezado.
\pagestyle{myheadings}
\markright{Universidad de la República \hfill Introducción a \LaTeX \hfill}
% ======================


%% ===== Ajuste layout pagina =====
% define el ancho del texto en la hoja
\setlength{\textwidth}{155mm}
% define el alto del texto en la hoja
\setlength{\textheight}{210mm}
% los márgenes pueden ser editador con
\oddsidemargin=-.25cm
%% ================================

% --- commandos definidos a gusto del usuario ---
\newcommand{\ds}{\displaystyle}
\def\x{{\bf x}}

%\newcommand{\subfigureautorefname}{\figureautorefname}
% -----------------
% =====================================================




% =====================================================
% ========  Aca comienza el cuerpo del texto ==========
%
\begin{document}
	
% Se renuevan comandos ya existentes de LaTeX como se desee, en este caso del nombre de tablas y figuras.	
\renewcommand{\tablename}{\bfseries Tabla} % Cambia nombre de tablas
\renewcommand{\figurename}{\bfseries Figura} % Cambia nombre de figuras 
%\newcommand{\subfigureautorefname}{\figureautorefname} % Para que al referenciar una subfigura aparezca "Figura"	
% Se refiere a las tablas y figuras con el comando \autoref{label} para que aparezca referenciado automáticamente el nombre de lo referenciado (Tabla o Figura) continuado por el número de la misma.	
%
% El comando \verb+maketitle+ sirve para escribir la cabecera con los datos del trabajo (título, autor y fecha).
\maketitle


% resumen
\begin{abstract}
En este párrafo se presenta un breve resumen del documento, es opcional y es particularmente usado en publicaciones en revistas. Para no colocarlo basta con comentar el párrafo. %
Generalmente se utilizan unas pocas líneas para describir el trabajo. %
En este documento presentaremos ejemplos de uso de \LaTeX~para escribir un documento y le daremos un formato tipo informe.
\end{abstract}

% índice
\tableofcontents

% para separar la carátula del texto introducimos un salto de pagina
\newpage

% definimos una sección con el comando section

\section{Introducción}
%
Considérese el siguiente problema: dada una grilla con n calles con dirección Este-Oeste y m calles con dirección Norte-Sur, en las cuales se representa el flujo de tránsito. Por cada intersección el flujo entra o sale, el mismo se mide en unidades de vehículos por hora (vph) promedio. Asumiendo que en cada intersección se cumplen las Ecuaciones de Balance, se busca determinar los vph que circulan en cada tramo de calle.
Generalizaremos este problema modelandolo de modo que conduzca al conjunto de soluciones de un sistema Ax = b, para luego describir las condiciones que debe cumplir A para que el sietema sea compatible.

\setlength{\parskip}{8mm}

Luego dado n = 8 y m = 7, buscaremos determinar los vph que circulan en cada tramo de la ciudad, para esto utilizaremos los archivos flujosh.mat, flujosv.mat y flujos.mat. Este nuevo problema deberá ser planteado mediante pivoteo parcial, comparando el resultado con la solución provista por Octave. Investigaremos si es posible obtener una solución alternativa utilizando el método de Jacobi, el de Gauss-Seidel, o alguna de las variantes de los mismos. Además deberemos resolver el corte de tránsito en una de las conexiones, explorando una solución al problema, buscando minimizar la circulación que se da por el puente.


\section{Marco Teórico}
%
En esta sección presentaríamos un marco teórico del contenido del trabajo, generalmente debe ser lo mínimo indispensable para obtener los resultados y concluir.

En nuestro caso serán secciones sobre mas temas de \LaTeX.

\subsection{Fuentes}

En \LaTeX se puede escribir con diferentes fuentes, sin embargo hay que saber controlarlas. En esta sección se muestran algunos ejemplos\footnote{En esta sección aprovechamos e introducimos el manejo de notas al pie.}.

%\times
Este es un ejemplo de otro tipo de fuente, en este caso caligráfica. Para poder escribir es necesario poner el paquete que controla las fuentes (fontenc) y el paquete de letra caligráfica (calligra).

Este es otro ejemplo de otro tipo de fuente, en este caso ``Carolingan Miniscules``. El paquete de letra es carolmin.

\normalfont
En la página ''http://www.tug.dk/FontCatalogue/alphfonts.html`` hay un catálogo muy extenso de fuentes para usar en \LaTeX. 

Para insertar hipervínculos podemos usar dos comandos en particular: href para esconder la ruta del hipervínculo bajo un texto

\href{http://www.tug.dk/FontCatalogue/alphfonts.html}{texto del Link}

o el comando url para mostrar directamente la ruta
\url{http://www.tug.dk/FontCatalogue/alphfonts.html}

\subsection{Tablas}
Los tablas son comunes en los trabajos científicos, sirven para representar datos de forma compacta. Vamos a mostrar un ejemplo de un tabla.

\subsubsection{Ejemplos}
\label{ej_fonts}
En el \autoref{cuadro_1} se muestra un ejemplo tomado de una tesis que trata sobre motores de combustión interna, particularmente los que siguen el ciclo de Otto. Tambien, se muestra como ejemplo un detalle que algunas veces puede pasar desapercibido, ya que no se utiliza comúnmente. Desde mi punto de vista es uno de los pocos inconveniente que tiene LaTeX que todavía no está solucionado. Me refiero a la nota al pie dentro de las tablas, que muchas veces son necesarias. Pero lamentablemente al día de hoy hay que ponerlas manualmente, en esta tabla se presenta un ejemplo de cómo se hace.

\begin{table}[!ht]
\centering
\caption[Aquí se escribe el comentario que se pone en el índice]{Este es el comentario de la tabla, donde se da una explicación detallada sobre el contenido del mismo.}
\label{cuadro_1}
\medskip 
\begin{tabular}{lc}
\hline \hline
 $r$, relación de compresión & $10$ \\
 $B$, diámetro interior del cilindro &  $79.5 \times 10^{-3}$ m\\
 $V_0$, volumen muerto de la cámara & $49.639 \times 10^{-6}$ m$^3$\\
 $T_w$, temperatura de la pared del cilindro & $600$ K\\
$T_1$, temperatura de entrada$^{\dag}$ & $333$ K\\
 $h$, coeficiente de transferencia de calor$^{\dag}$ & $1305$ W/m$^2$K\\
 $m$, masa de la mezcla de gases dentro del cilindro$^{\ddag}$ & $4.176\times 10^{-4}$ kg \\
\hline \hline
\end{tabular}
\flushleft
\footnotesize
$^{\dag}$ Sólo para TTF.
\\
$^{\ddag}$ Como condición inicial para la simulación numérica y fija para TTF.
\end{table}

Otro ejemplo de tabla en donde se pintan de un color diferente algunas columnas y se muestra cómo hacer columnas múltiples. El color en las tablas no se ve bien, pero si se compila en pdf se ve mejor.
\begin{table}[!ht]
\centering
\caption{Ejemplo de una tabla que muestra columnas múltiples y colores diferentes.}
\label{tab:cuadro_2}
\begin{tabular}{|c|cccc|}
\hline
 X/Y &  \multicolumn{4}{>{\columncolor[rgb]{0.8, 0.8, 0.8}}c|}{Población } \\ 
 Edad &  Montevideo &  Colonia & Salto & Rocha \\ 
\hline 
 20 &  23 &  34 &  56 & 87 \\ 
 25 &  22 &  56 &  76 & 23 \\ 
\hline 
\end{tabular}
\end{table}

Otro ejemplo de tabla es la \autoref{tab:presillas1}, en donde utilizamos el paquete \texttt{xcolor} para pintar intercaladamente las filas de la tabla.

\begin{table}[h]
	\centering
	\caption{Cantidad de presillas por barra.} \rowcolors{2}{}{gray!15} 
	\begin{tabular}{ccccccc}
		\textbf{Barra} & \textbf{L (m)} & \boldmath$\lambda_{max}$ & \boldmath$r_i$ \textbf{(cm)} & \boldmath$a_{max}$ \textbf{(cm)} & \textbf{nº presillas} & \boldmath$a_{final}$ \textbf{(cm)} \\
		\hline
		\hline
		\textbf{1} & 3.30  & 136   & 2.43  & 247.50 & 2     & 110.00 \\    \hline
		\textbf{2} & 3.30  & 136   & 2.43  & 247.50 & 2     & 110.00 \\    \hline
		\textbf{3} & 3.30  & 136   & 2.43  & 247.50 & 2     & 110.00 \\    \hline
		\textbf{4} & 3.30  & 136   & 2.43  & 247.50 & 2     & 110.00 \\    \hline
	\end{tabular}%
	\label{tab:presillas1}%
\end{table}%

En la \autoref{tab:conectividad} se presenta otro estilo de tablas, en donde se utilizan distintos espesores de línea además de columnas múltiples. Notar que al hacer columnas múltiples se especifica en la celda creada si se quiere tener lineas verticales a los lados.

\begin{table}[h]
	\centering
	\caption{Conectividad de elementos.} 
	\begin{tabular}{|c|c|c|}
		\toprule[0.8mm]                                                                 
		\multicolumn{3}{|c|}{Element's conectivity    }  \\  
		\midrule[0.5mm]                                  
		Element & Start & End \\ \midrule[0.5mm]                                                                                                                                                     
		1 &    1 &    2 \\
		2 &    2 &    3 \\
		3 &    3 &    4 \\
		4 &    4 &    5 \\
		\bottomrule[0.8mm]                                        
	\end{tabular}%
	\label{tab:conectividad}%
\end{table}%                                



\section{Desarrollo}
%
El desarrollo sería una de las partes centrales de un informe o artículo... en este caso presentaremos ejemplos de creación de fórmulas matemáticas con \LaTeX.

\subsection{Fórmulas matemáticas}
Hay varias maneras de insertar una fórmula matemática en un trabajo. Primero y más fácil es en la misma línea, como por ejemplo: $E=mc^{2}$. También se puede hacer en una línea aparte, sin numerar de esta forma $$E=mc^{2}$$ después se pueden poner en una línea aparte y numerarlas, de esta forma: 
\begin{equation}\label{ec:Ein}
  E=mc^{2}
\end{equation} %
%
es conveniente no dejar un renglón en blanco entre el texto y el entorno, así LaTeX dejará el espacio adecuado a cada fórmula. Se pueden hacer referencias a todas las etiquetas creadas, por ejemplo a la ecuación~\eqref{ec:Ein} o al \autoref{tab:cuadro_2}, incluso a la sección~\ref{ej_fonts}.


\subsection{Imágenes}
%
La \autoref{fig:optimo} es una figura de ejemplo, para un formato de la figura jpg, es necesario compilar el documento con el compilador PDFLaTeX, para formatos de figura eps, se puede compilar en Latex.
\begin{figure}[h!]
\centering
\includegraphics[width=1\textwidth]{figuras/optimo_primal}
\caption{Descripción de figura.}
% la etiqueta de las figuras siempre debe ir luego del caption
\label{fig:optimo}
\end{figure}


Se puede referenciar a cada figura del subfigure por ejemplo \autoref{fig:figmate}.

\begin{figure}[htb]
  \begin{subfigure}[t]{.6\linewidth}
	\centering
	\includegraphics[width=0.9\textwidth]{surf}
	\caption{Texto de imagen surf.}\label{fig:figsurf}
  \end{subfigure}%
  \begin{subfigure}[t]{.4\linewidth}
	\centering
	\includegraphics[width=0.9\textwidth]{mate}
	\caption{Esta es la imagen infrarroja de un mate junto a un termo donde se puede ver el reflejo de la radiación.}\label{fig:figmate}
  \end{subfigure}
\caption{Ejemplo de dos figuras usando subcaption.}
\label{fig:mapasej1_CECENR}
\end{figure}


\begin{minipage}[b]{0.45\textwidth}
También pueden ser incluidas figuras en formato pdf como por ejemplo en la Figura~\ref{fig:figmate} y \autoref{fig:figsurf}, que si vemos en el directorio figuras confirmamos que es un archivo pdf. El entorno minipage es útil para colocar figuras o texto en columnas, como se ve en este caso.  
    \end{minipage}
    %
    \hfil
    %    \pause
    %
    \begin{minipage}{0.45\textwidth}
        \begin{center}
            \includegraphics[width=1\textwidth]{chapa10}
        \end{center}
    \end{minipage}


\newpage

% ========================
\section{Conclusiones}
%
Aquí irían las conclusiones del artículo.  Tal vez se quieran mostrar conclusiones de forma esquemática como items... comoesto
%
\begin{itemize}
  \item Conclusión numero 1... 
  \item otra conclusión.
\end{itemize}

o también enumerándolas

\begin{enumerate}
  \item item primero...
  \item item segundo...
\end{enumerate}


\subsection{Citación de bibliografía}
%
Es posible citar artículos utilizados como referencia utilizando la función ``cite''. Por ejemplo citamos el libro \cite{lamport94} utilizando la etiqueta lamport94 que fue definida previamente al final del archivo latex como se puede ver en el código de este ejemplo. %
%
Se pueden citar tantos artículos como se quiera siempre que estén incluidos en el archivo \cite{otraetiqueta}. Existen otras formas mas complejas de citar en trabajos grandes como tesis o libros, utilizando archivos .bib . % ver http://en.wikibooks.org/wiki/LaTeX/Bibliography_Management
%


% =====================
\begin{thebibliography}{15}

\bibitem{lamport94}
  Leslie Lamport,
  \emph{\LaTeX: A Document Preparation System}.
  Addison Wesley, Massachusetts,
  2nd Edition,
  1994.

\bibitem{otraetiqueta}
  Otro autor,
  \emph{Título de artículo o libro}.
  Editorial o Journal,
  Edición o número de revista,
  Año.

\end{thebibliography}

% =====================

\end{document}
